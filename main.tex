% Template LaTeX file for LAC-25 papers
%
% To generate the correct references using BibTeX, run
%     latex, bibtex, latex, latex
% modified...
% - from DAFx-00 to DAFx-02 by Florian Keiler, 2002-07-08
% - from DAFx-02 to DAFx-03 by Gianpaolo Evangelista
% - from DAFx-05 to DAFx-06 by Vincent Verfaille, 2006-02-05
% - from DAFx-06 to DAFx-07 by Vincent Verfaille, 2007-01-05
%                          and Sylvain Marchand, 2007-01-31
% - from DAFx-07 to DAFx-08 by Henri Penttinen, 2007-12-12
%                          and Jyri Pakarinen 2008-01-28
% - from DAFx-08 to DAFx-09 by Giorgio Prandi, Fabio Antonacci 2008-10-03
% - from DAFx-09 to DAFx-10 by Hannes Pomberger 2010-02-01
% - from DAFx-10 to DAFx-12 by Jez Wells 2011
% - from DAFx-12 to DAFx-14 by Sascha Disch 2013
% - from DAFx-15 to DAFx-16 by Pavel Rajmic 2015
% - from DAFx-16 to IFC-18 by Romain Michon 2018
% - from IFC-18 to LAC-19 by Romain Michon 2019
% - from LAC-19 to LAC-25 by Pierre Lecomte 2024
%
% Template with hyper-references (links) active after conversion to pdf
% (with the distiller) or if compiled with pdflatex.
%
% 20060205: added package 'hypcap' to correct hyperlinks to figures and tables
%                      use of \papertitle and \paperauthorA, etc for same title in PDF and Metadata
%
% 1) Please compile using latex or pdflatex.
% 2) If using pdflatex, you need your figures in a file format other than eps! e.g. png or jpg is working
% 3) Please use "paperftitle" and "pdfauthor" definitions below

%------------------------------------------------------------------------------------------
%  !  !  !  !  !  !  !  !  !  !  !  ! user defined variables  !  !  !  !  !  !  !  !  !  !  !  !  !  !
% Please use these commands to define title and author(s) of the paper:
\def\papertitle { 
  Exploring WebAssembly as a \\
  Universal Distributable Binary \\
  for Headless Audio Plugins
}
\def\paperauthorA{ Joel A. Jaffe }
\def\paperauthorB{ Mahdi F. Ayman }
\def\paperauthorC{ Finn Buggy }
\def\paperauthorD{Author Four}

% Authors' affiliations have to be set below

%------------------------------------------------------------------------------------------
\documentclass[a4paper]{article}

\usepackage{style/LAC-25}
\usepackage[margin=2cm]{geometry}
\usepackage{tikz}
\usepackage{pgfplots}
\usepackage{float}
\usepackage{amsmath,amssymb,amsfonts,amsthm}
\usepackage{euscript}
\usepackage[utf8]{inputenc}
\usepackage[T1]{fontenc}
\usepackage{ifpdf}
\usepackage{color}
\usepackage{listings}
\definecolor{mygrey}{rgb}{0.96,0.96,0.96}
\lstset{
  tabsize=4,
  basicstyle=\ttfamily,
  backgroundcolor=\color{mygrey},
  captionpos=b,
  breaklines=true
}

\usepackage[english]{babel}
\usepackage{caption}
\usepackage{subfig, color}

\setcounter{page}{1}
\ninept

\usepackage{times}
% Saves a lot of ouptut space in PDF... after conversion with the distiller
% Delete if you cannot get PS fonts working on your system.

% pdf-tex settings: detect automatically if run by latex or pdflatex
\newif\ifpdf
\ifx\pdfoutput\relax
\else
   \ifcase\pdfoutput
      \pdffalse
   \else
      \pdftrue
\fi

\ifpdf % compiling with pdflatex
  \usepackage[pdftex,
    pdftitle={\papertitle},
    pdfauthor={\paperauthorA, \paperauthorB, \paperauthorC, \paperauthorD},
    colorlinks=false, % links are activated as colror boxes instead of color text
    bookmarksnumbered, % use section numbers with bookmarks
    pdfstartview=XYZ % start with zoom=100% instead of full screen; especially useful if working with a big screen :-)
  ]{hyperref}
  \pdfcompresslevel=9
  \usepackage{graphicx}
  \usepackage[figure,table]{hypcap}
\else % compiling with latex
  \usepackage[dvips]{epsfig,graphicx}
  \usepackage[dvips,
    colorlinks=false, % no color links
    bookmarksnumbered, % use section numbers with bookmarks
    pdfstartview=XYZ % start with zoom=100% instead of full screen
  ]{hyperref}
  % hyperrefs are active in the pdf file after conversion
  \usepackage[figure,table]{hypcap}
\fi

\title{\papertitle}

%-------------SINGLE-AUTHOR HEADER STARTS (uncomment below if your paper has a single author)-----------------------
\affiliation{
\paperauthorA}
{\href{https://www.mat.ucsb.edu/}{Department of Media Arts \& Technology} \\
\href{https://www.ucsb.edu}{University of California Santa Barbara, USA} \\
{\tt \href{mailto:joel@jaffesd.com}{joel@jaffesd.com}}
}
%-----------------------------------SINGLE-AUTHOR HEADER ENDS------------------------------------------------------

%---------------TWO-AUTHOR HEADER STARTS (uncomment below if your paper has two authors)-----------------------
% \twoaffiliations{
% \paperauthorA \,\sthanks{This work was supported by the XYZ Foundation}}
% {\href{https://ccrma.stanford.edu}{CCRMA} \\ Stanford University, USA \\
% {\tt \href{mailto:lac@ccrma.stanford.edu}{lac@ccrma.stanford.edu}}
% }
% {\paperauthorB \,\sthanks{This guy is a very good fellow}}
% {\href{http://www.musikwissenschaft.uni-mainz.de/Musikinformatik/}{Johannes Gutenberg University (JGU)} \\  Mainz, Germany\\
% {\tt \href{mailto:lac@uni-mainz.de}{lac@uni-mainz.de}}
% }
%-------------------------------------TWO-AUTHOR HEADER ENDS------------------------------------------------------

%---------------THREE-AUTHOR HEADER STARTS (uncomment below if your paper has three authors)-----------------------
% \threeaffiliations 
% \paperauthorA \,\sthanks{This work was supported by the XYZ Foundation}}
% {\href{https://ccrma.stanford.edu}{CCRMA} \\ Stanford University, USA \\
% {\tt \href{mailto:lac@ccrma.stanford.edu}{lac@ccrma.stanford.edu}}
% }
% {\paperauthorB \,\sthanks{This guy is a very good fellow}}
% {\href{http://www.musikwissenschaft.uni-mainz.de/Musikinformatik/}{Johannes Gutenberg University (JGU)} \\  Mainz, Germany\\
% {\tt \href{mailto:lac@uni-mainz.de}{lac@uni-mainz.de}}
% }
% {\paperauthorC \,\sthanks{Illustrious contributor}}
% {\href{https://lmfa.ec-lyon.fr/}{LMFA} \\ Ecully, France \\
% {\tt \href{mailto:lac@ec-lyon.fr}{lac@ec-lyon.fr}}
% }
%-------------------------------------THREE-AUTHOR HEADER ENDS------------------------------------------------------

%----------------FOUR-AUTHOR HEADER STARTS (uncomment below if your paper has four authors)-----------------------
% \fouraffiliations{
% \paperauthorA \,\sthanks{This work was supported by the XYZ Foundation}}
% {\href{https://ccrma.stanford.edu}{CCRMA} \\ Stanford University, USA \\
% {\tt \href{mailto:lac@ccrma.stanford.edu}{lac@ccrma.stanford.edu}}
% }
% {\paperauthorB \,\sthanks{This guy is a very good fellow}}
% {\href{http://www.musikwissenschaft.uni-mainz.de/Musikinformatik/}{Johannes Gutenberg University (JGU)} \\  Mainz, Germany\\
% {\tt \href{mailto:lac@uni-mainz.de}{lac@uni-mainz.de}}
% }
% {\paperauthorC \,\sthanks{Illustrious contributor}}
% {\href{https://lmfa.ec-lyon.fr/}{LMFA} \\ Ecully, France \\
% {\tt \href{mailto:lac@ec-lyon.fr}{lac@ec-lyon.fr}}
% }
% {\paperauthorD \,\sthanks{Thanks to the predessors for the templates}}
% {\href{https://www.univ-st-etienne.fr}{Jean Monnet University (UJM)} \\
% Saint-Etienne, France \\
% {\tt \href{mailto:lac@univ-st-etienne.fr}{lac@univ-st-etienne.fr}}
% }
%-------------------------------------FOUR-AUTHOR HEADER ENDS------------------------------------------------------

\begin{document}
% more pdf-tex settings:
\ifpdf % used graphic file format for pdflatex
  \DeclareGraphicsExtensions{.png,.jpg,.pdf}
\else  % used graphic file format for latex
  \DeclareGraphicsExtensions{.eps}
\fi

\maketitle

\begin{abstract}

Originally developed to enhance the performance of web applications, 
WebAssembly has gained recognition as a portable, 
secure and efficient runtime environment 
that provides ``write once, run anywhere'' deployment 
\& maintenance of software applications 
in a large variety of programming languages. 
While research into WebAssembly-based audio plugins is mature, 
prior work has mainly focused on their deployment in the browser. 
This project explores the performance of WebAssembly audio plugins 
deployed on a wider variety of targets, 
with an emphasis on IoMusT applications. 
To investigate the feasibility of WebAssembly as a universal 
distributable binary for heterogenous targets,
four representative audio processing routines are 
compiled to WebAssembly with a consistent binary interface, 
and three methods are benchmarked for running 
the programs on a resource contained target (Daisy Seed). 
Preliminary results showcase that two of the methods provide 
performance tradeoffs acceptable for real-time processing,
and that performance may be improved further by using 
WebAssembly's capabilities for importing math functions from plugin hosts.

% Abstract must:
% \begin{itemize}
% \item describe general problem and why it is important.
% \item describe your insights/opportunity to fix problem.
% \item describe your approach/methods
% \item describe your evaluation results
% \item describe your contributions.
% \end{itemize}

\end{abstract}

\section{ Introduction }
\label{sec:intro}

The landscape of audio software in 2026 is a notably diverse one.
Targets include:

\begin{itemize}
\item standalone applications spanning 
various operating systems and computer architectures
\item numerous audio plugin architectures, 
targeting primarily DAWs and game engines (VST, AUv3, LV2)
\item web audio
\item embedded systems
\item mobile applications
\end{itemize}

Interoperability between these different environments
can be desirable from both end-user and developer perspectives.
For users, interoperability allows for a portability of sounds 
between use cases. In example, an electric instrumentalist 
may want the same effects chain they use in a DAW to also be usable
in a live performance context, ideally on hardware that is tailored 
to that environment. For developers, deployment
to multiple targets from a single codebase and build system 
enables easier development and maintenance of audio DSP code.

In order to facilitate deployment to multiple of the 
above mentioned targets from a single codebase,
\textit{intermediary plugin frameworks} have become popular 
among audio developers\cite{2018_13}. 
The most notable of these frameworks is JUCE, 
which has become the industry standard. 
JUCE provides a C++ API by which the same audio processing
and user interface code can be compiled for multiple targets, 
including cross-platform standalone audio applications, 
DAW \& game engine plugins, and mobile applications.
An alternative intermediary plugin framework, iPlug2, 
adds Web Audio as a target, and is liberally licensed, 
but is also implemented as a C++ API.

Efforts have been made to provide alternative strategies 
to audio developers that do not rely on C++ tooling, 
and can target a wider variety of environments.

One approach is that of textual domain-specific languages (DSL)
for audio programming, with CSound, FAUST, and CMajor
being the most notable examples. 
The primary advantage of these languages is that they 
enable the developer to focus strictly on audio processing
logic, without handling memory management 
and other low-level programming tasks.
They also often have the ability to target multiple
deployment environments by exporting generated code.
Some downsides are that they require programmers to learn
a new language, and that they may constrain the developer 
to a particular paradigm of signal processing.

Another approach is visual programming languages for audio, 
such as Max/MSP, Gen\~, and PureData. These languages
work in a data flow paradigm using boxes (small programs) 
connected by data flow lines, an interface that mirrors 
the composition of hardware signal processing devices
via patch cables, which is often familiar to musicians and 
sounds engineers. Like textual DSLs for audio, these languages
are geared towards a particular paradigm of signal processing 
that makes difficult many techniques developed in recent years,
such as neural audio inference engines. Like textual DSLs for audio,
these visual languages are often also capable of 
exporting code to multiple targets.

In the case of both textual DSLs and visual programming languages,
their ability to target multiple deployment environments may be visualized
as seen in Figure~\ref{fig1}. They are a single-input, multiple output systems.

\begin{figure}[ht]
\centering
\begin{tikzpicture}
\node[draw,rectangle] (dsl) at (0,0) {Audio DSL};
\node[draw,rectangle] (t1) at (3,1) {Target 1};
\node[draw,rectangle] (t2) at (3,0) {Target 2};
\node[draw,rectangle] (t3) at (3,-1) {Target 3};
\draw[->] (dsl) -- (t1);
\draw[->] (dsl) -- (t2);
\draw[->] (dsl) -- (t3);
\end{tikzpicture}
\caption{\label{fig1}{\it Graphical representation
of an audio DSL targeting multiple 
deployment environments.}}
\end{figure}

A more flexible approach is a solution that can receive input
in the form of source code from myriad programming languages,
and also target multiple deployment environments. 
Traditionally this may look like Figure~\ref{fig2}, 
depicting how each programming language must have its own
toolchain to target each deployment environment. 

\begin{figure}[H]
\centering
\begin{tikzpicture}
\node[draw,rectangle] (l1) at (0,1) {Language 1};
\node[draw,rectangle] (l2) at (0,0) {Language 2};
\node[draw,rectangle] (l3) at (0,-1) {Language 3};
\node[draw,rectangle] (t1) at (3,1) {Target 1};
\node[draw,rectangle] (t2) at (3,0) {Target 2};
\node[draw,rectangle] (t3) at (3,-1) {Target 3};
\foreach \i in {1,2,3} {
\foreach \j in {1,2,3} {
\draw[->] (l\i) -- (t\j);
}
}
\end{tikzpicture}
\caption{\label{fig2}{\it Graphical representation 
of per-language, per-target toolchains.}}
\end{figure}

By utilizing an intermediate representation,
the complexity of this system may be reduced.
A single runtime for the intermediate format on any target
enables all programming languages that can compile to the
representation to immediately target that environment.
Conversely, any programming language that can compile to the
intermediate representation may target any environment that
has a runtime for that representation. 
This is visualized in Figure~\ref{fig3}.

\begin{figure}[H]
\center
\begin{tikzpicture}
\node[draw,rectangle] (l1) at (0,1) {Language 1};
\node[draw,rectangle] (l2) at (0,0) {Language 2};
\node[draw,rectangle] (l3) at (0,-1) {Language 3};
\node[draw,rectangle] (ir) at (3,0) {Intermediate Representation};
\node[draw,rectangle] (t1) at (6,1) {Target 1};
\node[draw,rectangle] (t2) at (6,0) {Target 2};
\node[draw,rectangle] (t3) at (6,-1) {Target 3};
\draw[->] (l1) -- (ir);
\draw[->] (l2) -- (ir);
\draw[->] (l3) -- (ir);
\draw[->] (ir) -- (t1);
\draw[->] (ir) -- (t2);
\draw[->] (ir) -- (t3);
\end{tikzpicture}
\caption{\label{fig3}{\it Graphical representation
of multiple languages targeting an 
intermediate representation, 
which supports multiple targets.}}
\end{figure}

\section{ Related Work }
\label{ssec:Related Work}

Maybe mention Sonical.

Definitely mention AudioReach.

\subsection{ Domain-Specific Programming for Audio DSP (FAUST and Gen\~~) }
FAUST[], a domain specific programming language for real-time DSP. 
FAUST affords the programmer the ability to define audio processing 
without having to manually manage memory, 
and to target multiple different deployments from the same code. 
The downside of the bargain is that the programmer must learn Faust.

Note: Figure 17 of 
"FAUST Domain Specific Audio DSP Language Compiled to WebAssembly" 
shows the performance of FAUST->C->emcc->wasm roughly equal to FAUST->wasm.

A node-oriented option is Gen\~~. 
Gen\~~ affords the ability to create efficient, 
low-level[] audio processing without having to write code at all, 
yet also exposes a textual code interface for advanced users. 
Like FAUST, Gen also deploys to many targets, by generating code.

\subsection{ Audio Plugin Standards }
\subsection{ Cool Third Thing }

\section{ Methods }

To investigate the feasibility of WebAssembly as a universal 
distributable binary for heterogenous targets,
four representative audio processing routines are 
compiled to WebAssembly with a 
consistent binary interface, 
and three methods are benchmarked for running 
the programs on a resource contained target 
(Daisy Seed). 

The Daisy Seed is a widely popular embedded platform, 
similar to the Arduino ecosystem but with a focus on
audio. It is available in both development and manufacturer 
oriented packages, making it a popular choice across hobbyist, 
education, and industry use cases. 
It supports 32-bit processing at a 480MHz clock rate
and can be programmed from various languages.

To benchmark the performance of compiled WebAssembly plugins,
four audio processing algorithms are compiled to both WebAssembly
and native ARM binary, and their processing speeds are compared.

The algorithms are:

\begin{itemize}
  \item A FAUST spring reverb exported to C++
  \item An lightweight implementation of Neural Amp Modeler
  \item A virtual analog model of a Klon Centaur overdrive circuit using WDFs
  \item A fourth thing I haven't decided on yet
\end{itemize}

Four WebAssembly runtimes are compared. 
The runtimes are:

\begin{itemize}
  \item Wasmi, a WebAssembly interpreter written in Rust
  \item Web Assembly Micro Runtime,
  \item De- and re- compilation of WebAssembly to C to native ARM binary
\end{itemize}

\section{ Results }

Preliminary results:

\begin{table}[ht]
\caption{\itshape Performance comparison 
of different runtimes across algorithms.}
\centering
\begin{tabular}{|l|c|c|c|c|}
  \hline
  Algorithm & Native & WAMR AOT & Wasm2c & Wasmi \\
  \hline
  FAUST Spring Reverb & 1.0x & 0.970x & 0.852x & 0.081x \\
Neural Amp Modeler & XX & XX & XX & XX \\
Klon Centaur Overdrive & XX & XX & XX & XX \\
Fourth Algorithm & XX & XX & XX & XX \\
  \hline
\end{tabular}
\label{tab:performance}
\end{table}

\begin{figure*}[ht]
\centering
\begin{tikzpicture}
\begin{axis}[
    ybar,
    bar width=0.67cm,
    nodes near coords,
    enlarge x limits=0.2,
    ymin=0.0,
    ylabel={Relative Performance},
    xlabel={Runtime},
    symbolic x coords={WAMR AOT, Wasm2c, Wasmi},
    xtick=data,
    legend style={at={(0.5, -0.15)},anchor=north},
    legend columns=2,
    width=\textwidth,
    height=0.8\textwidth,
]
\addplot coordinates {(WAMR AOT,0.97) (Wasm2c,0.852) (Wasmi,0.081)};
\addplot coordinates {(WAMR AOT,0.85) (Wasm2c,0.72) (Wasmi,0.03)};
\addplot coordinates {(WAMR AOT,0.92) (Wasm2c,0.81) (Wasmi,0.06)};
\addplot coordinates {(WAMR AOT,0.89) (Wasm2c,0.78) (Wasmi,0.05)};
\legend{FAUST Spring Reverb, Neural Amp Modeler, Klon Centaur Overdrive, Fourth Algorithm}
\end{axis}
\end{tikzpicture}
\caption{Bar graph of performance comparison 
for all algorithms, with dummy data for now.}
\end{figure*}


\section{ Evaluation }

\section{ Conclusion }

\section{ Acknowledgments }

All figures should be centered on the column (or page, if the figure spans both
columns). Figure captions (in italic) should follow each figure and have the
format given in Figure~\ref{fft_plot}. Vectorial figures are preferred (e.g.,
Postscript, PDF, etc.). Also, in order to provide a better readability, figure
text font size should be at least identical to footnote font size. If bitmap
figures are used, please make sure that the resolution is enough for print
quality. Figure~\ref{ftt_plot2} illustrates an example of a figure spanning two
columns.

\begin{figure}[ht]
\centerline{\includegraphics[scale=0.5]{figures/ping}}
\caption{\label{fft_plot}{\it Ping.}}
\end{figure}

\begin{figure*}[ht]
\center
\includegraphics[width=5in]{figures/TwoColumnSine2/TwoColumnSine2}
\caption{\label{ftt_plot2}{\it A figure spanning two columns, as mentioned in
Sec. \ref{ssec:figures}.}}
\end{figure*}

As for figures, all tables should be centered on the column (or page, if the
table spans both columns). Table captions should be in italic, precede each
table and have the format given in Table~\ref{tab:example}.

\begin{table}[ht]
  \caption{\itshape Basic trigonometric values.}
	\centering
	\begin{tabular}{|c|c|}
		\hline
		$\mathrm{angle}\,(\theta, \mathrm{rad})$ & $\sin \theta$ \\\hline
		$\frac{\pi}{2}$ & $1$ \\
		$\pi$ & $0$ \\
		$\frac{3\pi}{2}$ & $-1$ \\
		$2\pi$ & $0$ \\\hline
	\end{tabular}
	%
	\label{tab:example}
\end{table}

\begin{table*}[ht]
  \caption{{\it Basic trigonometric values, spanning two columns.}}
	\centering
  \begin{tabular}{|c|c|c|c|c|c|c|}\hline
    $\mathrm{angle}\, (\theta, \mathrm{rad})$ & $\sin \theta$ & $\cos \theta $ & $(\sin \theta)/2 $ & $(\cos \theta) /2 $ & $(\sin \theta)/3 $ & $(\cos \theta)/3$    \\\hline
    $\frac{\pi}{2}$ & $1$ & $0$ & $1/2$ & $0$ & $1/3$ & $0$ \\
    $\pi$ & $0$ & $-1$ & $0$ & $-1/2$ & $0$ & $-1/3$\\
    $\frac{3\pi}{2}$ & $-1$ & $0$ & $-1/2$ & $0$ & $-1/3$ & $0$ \\
    $2\pi$ & $0$ & $1$ & $0$ & $1/2$ & $0$ & $1/3$ \\\hline
 \end{tabular}
	%
  \label{tab:example2}
\end{table*}

\subsection{Equations}

Equations should be placed on separate lines and numbered:

\begin{equation}
	y(n)=b_0x(n)-a_1y(n-1)
	\label{eq1}
	\end{equation}
	where equation (\ref{eq1}) is a one pole filter with frequency response:
	\begin{equation}
	H(e^{j \omega T}) = \frac{b_0}{1+a_1e^{-j \omega T}}
	\label{eq2}
\end{equation}

\subsection{Code}

Code can be listed in a block:

\begin{lstlisting}
  int foo = 0;
\end{lstlisting}
\noindent
or directly in-lined in the body of the text: \lstinline{int foo = 1;}.

\subsection{Page Numbers}

Page numbers will be added to the document in the post-processing stage, so
{\em please leave the numbering as is} (no numbers).


\subsection{References}

The references will be numbered in order of appearance \cite{2018_13},
\cite{Spa72}, \cite{MosWal64} and \cite{Kay86}. Please avoid listing
references that do not appear in the text.

\subsubsection{Reference Format}

The reference format is the standard IEEE one. We recommend to use BibTeX to
create the reference list.

\section{Conclusions}

This template can be found on the conference website. For changing the number
of author affiliations (1 to 4), uncomment the corresponding regions in the
template \texttt{tex} file. Please, submit full-length papers (4 to 8 pages
for full papers and 2 to 4 pages for poster papers) and keep the paper size to
letter (don't change to A4). Submission is fully electronic and automated 
through the Conference Web Submission System. DO NOT send us papers directly by 
e-mail.

\section{Acknowledgments}

Many thanks to the great number of anonymous reviewers!

%\newpage
\nocite{*}
\bibliographystyle{style/IEEEbib}
\bibliography{style/LAC-25} % requires file lac-25.bib

\section{Appendix: Margin Check}

This section shows the column margins for the text.

Lorem ipsum dolor sit amet, consectetur adipisici elit, sed eiusmod tempor
incidunt ut labore et dolore magna aliqua. Ut enim ad minim veniam, quis
nostrud exercitation ullamco laboris nisi ut aliquid ex ea commodi consequat.
Quis aute iure reprehenderit in voluptate velit esse cillum dolore eu fugiat
nulla pariatur. Excepteur sint obcaecat cupiditat non proident, sunt in culpa
qui officia deserunt mollit anim id est laborum.

Duis autem vel eum iriure dolor in hendrerit in vulputate velit esse molestie
consequat, vel illum dolore eu feugiat nulla facilisis at vero eros et accumsan
et iusto odio dignissim qui blandit praesent luptatum zzril delenit augue duis
dolore te feugait nulla facilisi. Lorem ipsum dolor sit amet, consectetuer
adipiscing elit, sed diam nonummy nibh euismod tincidunt ut laoreet dolore
magna aliquam erat volutpat.

Ut wisi enim ad minim veniam, quis nostrud exerci tation ullamcorper suscipit
lobortis nisl ut aliquip ex ea commodo consequat. Duis autem vel eum iriure
dolor in hendrerit in vulputate velit esse molestie consequat, vel illum dolore
eu feugiat nulla facilisis at vero eros et accumsan et iusto odio dignissim qui
blandit praesent luptatum zzril delenit augue duis dolore te feugait nulla
facilisi.

Nam liber tempor cum soluta nobis eleifend option congue nihil imperdiet doming
id quod mazim placerat facer possim assum. Lorem ipsum dolor sit amet,
consectetuer adipiscing elit, sed diam nonummy nibh euismod tincidunt ut
laoreet dolore magna aliquam erat volutpat. Ut wisi enim ad minim veniam, quis
nostrud exerci tation ullamcorper suscipit lobortis nisl ut aliquip ex ea
commodo consequat.

Duis autem vel eum iriure dolor in hendrerit in vulputate velit esse molestie
consequat, vel illum dolore eu feugiat nulla facilisis.

At vero eos et accusam et justo duo dolores et ea rebum. Stet clita kasd
gubergren, no sea takimata sanctus est Lorem ipsum dolor sit amet. Lorem ipsum
dolor sit amet, consetetur sadipscing elitr, sed diam nonumy eirmod tempor
invidunt ut labore et dolore magna aliquyam erat, sed diam voluptua. At vero
eos et accusam et justo duo dolores et ea rebum. Stet clita kasd gubergren, no
sea takimata sanctus est Lorem ipsum dolor sit amet. Lorem ipsum dolor sit
amet, consetetur sadipscing elitr, At accusam aliquyam diam diam dolore dolores
duo eirmod eos erat, et nonumy sed tempor et et invidunt justo labore Stet
clita ea et gubergren, kasd magna no rebum. sanctus sea sed takimata ut vero
voluptua. est Lorem ipsum dolor sit amet. Lorem ipsum dolor sit amet,
consetetur sadipscing elitr, sed diam nonumy eirmod tempor invidunt ut labore
et dolore magna aliquyam erat.

Consetetur sadipscing elitr, sed diam nonumy eirmod tempor invidunt ut labore
et dolore magna aliquyam erat, sed diam voluptua. At vero eos et accusam et
justo duo dolores et ea rebum. Stet clita kasd gubergren, no sea takimata
sanctus est Lorem ipsum dolor sit amet. Lorem ipsum dolor sit amet, consetetur
sadipscing elitr, sed diam nonumy eirmod tempor invidunt ut labore et dolore
magna aliquyam erat, sed diam voluptua. At vero eos et accusam et justo duo
dolores et ea rebum. Stet clita kasd gubergren, no sea takimata sanctus est
Lorem ipsum dolor sit amet. Lorem ipsum dolor sit amet, consetetur sadipscing
elitr, sed diam nonumy eirmod tempor invidunt ut labore et dolore magna aliquyam
erat, sed diam voluptua. At vero eos et accusam et justo duo dolores et ea
rebum. Stet clita kasd gubergren, no sea takimata sanctus est Lorem ipsum dolor
sit amet.

Lorem ipsum dolor sit amet, consectetur adipisici elit, sed eiusmod tempor
incidunt ut labore et dolore magna aliqua. Ut enim ad minim veniam, quis
nostrud exercitation ullamco laboris nisi ut aliquid ex ea commodi consequat.
Quis aute iure reprehenderit in voluptate velit esse cillum dolore eu fugiat
nulla pariatur. Excepteur sint obcaecat cupiditat non proident, sunt in culpa
qui officia deserunt mollit anim id est laborum.


Duis autem vel eum iriure dolor in hendrerit in vulputate velit esse molestie
consequat, vel illum dolore eu feugiat nulla facilisis at vero eros et accumsan
et iusto odio dignissim qui blandit praesent luptatum zzril delenit augue duis
dolore te feugait nulla facilisi. Lorem ipsum dolor sit amet, consectetuer
adipiscing elit, sed diam nonummy nibh euismod tincidunt ut laoreet dolore
magna aliquam erat volutpat.

Ut wisi enim ad minim veniam, quis nostrud exerci tation ullamcorper suscipit
lobortis nisl ut aliquip ex ea commodo consequat. Duis autem vel eum iriure
dolor in hendrerit in vulputate velit esse molestie consequat, vel illum dolore
eu feugiat nulla facilisis at vero eros et accumsan et iusto odio dignissim qui
blandit praesent luptatum zzril delenit augue duis dolore te feugait nulla
facilisi.

Nam liber tempor cum soluta nobis eleifend option congue nihil imperdiet doming
id quod mazim placerat facer possim assum. Lorem ipsum dolor sit amet,
consectetuer adipiscing elit, sed diam nonummy nibh euismod tincidunt ut
laoreet dolore magna aliquam erat volutpat. Ut wisi enim ad minim veniam, quis
nostrud exerci tation ullamcorper suscipit lobortis nisl ut aliquip ex ea
commodo consequat.

Duis autem vel eum iriure dolor in hendrerit in vulputate velit esse molestie
consequat, vel illum dolore eu feugiat nulla facilisis.

At vero eos et accusam et justo duo dolores et ea rebum. Stet clita kasd
gubergren, no sea takimata sanctus est Lorem ipsum dolor sit amet. Lorem ipsum
dolor sit amet, consetetur sadipscing elitr, sed diam nonumy eirmod tempor
invidunt ut labore et dolore magna aliquyam erat, sed diam voluptua. At vero
eos et accusam et justo duo dolores et ea rebum. Stet clita kasd gubergren, no
sea takimata sanctus est Lorem ipsum dolor sit amet. Lorem ipsum dolor sit amet,
consetetur sadipscing elitr, At accusam aliquyam diam diam dolore dolores duo
eirmod eos erat, et nonumy sed tempor et et invidunt justo labore Stet clita ea
et gubergren, kasd magna no rebum. sanctus sea sed takimata ut vero voluptua.
est Lorem ipsum dolor sit amet. Lorem ipsum dolor sit amet, consetetur
sadipscing elitr, sed diam nonumy eirmod tempor invidunt ut labore et dolore
magna aliquyam erat.

Consetetur sadipscing elitr, sed diam nonumy eirmod tempor invidunt ut labore
et dolore magna aliquyam erat, sed diam voluptua. At vero eos et accusam et
justo duo dolores et ea rebum. Stet clita kasd gubergren, no sea takimata
sanctus est Lorem ipsum dolor sit amet. Lorem ipsum dolor sit amet, consetetur
sadipscing elitr, sed diam nonumy eirmod tempor invidunt ut labore et dolore
magna aliquyam erat, sed diam voluptua. At vero eos et accusam et justo duo
dolores et ea rebum. Stet clita kasd gubergren, no sea takimata sanctus est
Lorem ipsum dolor sit amet.

\end{document}
